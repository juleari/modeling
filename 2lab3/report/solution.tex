\section{Постановка задачи}
    Найти корреляцию между вошедшими в метро пассажирами -- $x_1$ (Таблица 1) и прокомпостированными билетами -- $x_3$ (Таблица 2).

\section{Данные метрополитена} % (fold)
        \begin{table}[h!]%{|c|c|c|c|c|c|c|} \hline
        \centering
            \begin{tabular}{|c|c|c|c|c|c|c|} \hline
                     &   A0  &   A1  &   B0  &   B1  &   C0  &   C1  \\ \hline
            Январь   & 16551 & 30746 & 32822 & 21002 & 17084 & 4544  \\ \hline
            Февраль  & 16810 & 22558 & 25314 & 40022 & 29096 & 17519 \\ \hline
            Март     & 14434 & 28001 & 36918 & 35118 & 38639 & 38841 \\ \hline
            Апрель   & 20891 & 32958 & 46677 & 20283 & 23690 & 37324 \\ \hline
            Май      & 13773 & 28277 & 16909 & 41746 & 29087 & 16717 \\ \hline
            Июнь     & 14739 & 36763 & 21889 & 40458 & 21993 & 40099 \\ \hline
            Июль     & 24713 & 34650 & 34998 & 19478 & 30082 & 42244 \\ \hline
            Август   & 10127 & 33590 & 23285 & 22974 & 18776 & 22099 \\ \hline
            Сентябрь & 14689 & 12239 & 21561 & 25348 & 34808 & 40895 \\ \hline
            Октябрь  & 13047 & 35848 & 37778 & 25336 & 26192 & 17519 \\ \hline
            Ноябрь   & 16487 & 38451 & 29376 & 23743 & 18230 & 38841 \\ \hline
            Декабрь  & 14345 & 18573 & 32822 & 29751 & 37085 & 37324 \\ \hline
            \end{tabular}
        \caption{Количество вошедших в метро $(x_1)$}
        %\label{tab:tabx1}
        \end{table}

        \begin{table}[h!]%{|c|c|c|c|c|c|c|} \hline
        \centering
            \begin{tabular}{|c|c|c|c|c|c|c|} \hline
                    &   A0  &   A1  &   B0  &   B1  &   C0  &  C1   \\ \hline
            Январь  & 14899 & 27320 & 29553 & 18793 & 15365 & 3118  \\ \hline
            Февраль & 14292 & 20155 & 22567 & 35436 & 25876 & 16162 \\ \hline
            Март    & 13046 & 24916 & 32720 & 31145 & 34226 & 34819 \\ \hline
            Апрель  & 18696 & 29255 & 41259 & 18164 & 21145 & 33492 \\ \hline
            Май     & 12468 & 25159 & 15212 & 36944 & 25868 & 15461 \\ \hline
            Июнь    & 13313 & 32398 & 19569 & 35817 & 20494 & 35920 \\ \hline
            Июль    & 22040 & 30735 & 31040 & 17460 & 26738 & 37797 \\ \hline
            Август  & 9278  & 29808 & 20791 & 21353 & 17263 & 20170 \\ \hline
            Сентябрь& 13269 & 11126 & 19282 & 23430 & 31290 & 36617 \\ \hline
            Октябрь & 11833 & 31784 & 33472 & 22586 & 23751 & 16162 \\ \hline
            Ноябрь  & 14843 & 34061 & 26120 & 22025 & 16784 & 34819 \\ \hline
            Декабрь & 12968 & 16668 & 29553 & 27282 & 33283 & 33492 \\ \hline
            \end{tabular}
        \caption{Количество прокомпостированных билетов $(x_3)$}
        %\label{tab:tabx1}
        \end{table}

\section{Коэффициент коррелляции Пирсона}
    Пусть даны две выборки $x, y$.
    Тогда коэффициент корреляции Пирсона рассчитывается по следующей формуле:
    \begin{gather}
        r_{xy} = \frac{\sum\limits_{i=1}^n(x_i - \overline{x})(y_i - \overline{y})}{\sqrt{\sum\limits_{i=1}^n(x_i - \overline{x})^2\sum\limits_{i=1}^n(y_i - \overline{y})^2}} = \frac{cov(x, y)}{\sqrt{s_x^2s_y^2}}
    \end{gather}
    где $\overline{x}, \overline{y}$ -- выборочные средние, $s_x^2, s_y^2$ -- выборочные дисперсии, $r_{xy} \in [-1, 1]$.

    \begin{itemize}
        \item $|r_{xy}| = 1$ -- $x, y$ линейно зависимы
        \item $|r_{xy}| = 0$ -- $x, y$ линейно независимы
    \end{itemize}
\section{Ранговый коэффициент коррелляции Пирсона}
    Пусть даны две выборки $x, y$.
    По значениям получаем ранги $r_x, r_y$.
    Тогда ранговый коэффициент корреляции Пирсона рассчитывается по следующей формуле:
    \begin{gather}
        r_{xy} = \sum_{i=1}^n\frac{(x_i - y_i)^2}{y_i}
    \end{gather}
\section{Ранговый коэффициент коррелляции Спирмана}
    Пусть даны две выборки $x, y$.
    По значениям получаем ранги $r_x, r_y$.
    Тогда ранговый коэффициент корреляции Пирсона рассчитывается по следующей формуле:
    \begin{gather}
        r_{xy} = 1 - \frac6{n*(n^2 - 1)}\sum_{i=1}^n(x_i - y_i)^2
    \end{gather}
\newline

%\newpage

\section{Код программы}
    \lstinputlisting[language=Python]{../py/lab.py}

\section{Результаты}
    \lstinputlisting{../results/retdata}

\section{Оценка результатов}
    Линейный коэффициент Пирсона показал близкие к единице результаты на всех шести наборах данных.
    Из этого можно сделать вывод о корреляции.
    
    Для проверки значений рангового коэффициента Пирсона используются табличные значения ранговых коэффициентов при уровне значимости $0.01$ и размере выборки $12$.
    Чтобы счесть выборки коррелирующими необходимо, чтобы полученное значение попадало в промежуток $[-0.708, 0.708]$
    Как видно из результатов -- все полученные значения попадают в заданную область.

    Значимость рангового коэффициента Спирмана приблизительно равна $14.03$. 
    Табличное значение при уровне значимости $0.01$ равно $3.106$.
    Значит коэффициент ранговой корреляции Спирмена статистически значим.
    Доверительный интервал: $[0.761, 1.190]$.
    Ранговый коэффициент Спирмана на всех наборах показал близкие к единице результаты, что также говорит о линейной зависимости данных.